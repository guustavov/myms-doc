\chapter{Trabalhos Relacionados}
\label{relacionados}

Este capítulo é dedicado a apresentar trabalhos desenvolvidos na área de segurança computacional, com ênfase naqueles que utilizam-se de técnica de inteligência artificial para realizar a classificação de eventos de redes de computadores visando a detecção de atividades intrusivas. O intuito é apresentar um histórico e trabalhos situados no estado da arte, com vistas a identificar lacunas que podem ser trabalhadas, servindo também como fator motivacional para o desenvolvimento do presente trabalho.


\section{Histórico}

Conforme mencionado de forma breve no capítulo \ref{Cap:fundamentacao}, sistemas de detecção de intrusão podem ser construídos de diferentes modos, observando-se a  estratégia de análise dos eventos ocorridos. A fim de realizar a tarefa de detecção de intrusão, podem ser consideradas regras preestabelecidas que mapeiam o comportamento de uma atividade ilícita conhecida, abordagem conhecida como detecção por abuso (ou por assinatura). Outra abordagem amplamente utilizada é a detecção por anomalia, que possui como ideia principal o mapeamento de atividades normais. Sendo assim, toda atividade que fugir do padrão determinado como normal, possui grande probabilidade de tratar-se de um ataque.

O presente trabalho busca a abstração e generalização de ambos os tipos de eventos (normais e anômalos), através da criação de modelos de redes neurais artificiais. Por não utilizar base de regras que definem características explícitas de um ataque, mas sim uma base de exemplos de atributos extraídos de conexões entre computadores, considera-se uma abordagem que realiza detecção por anomalia.

Ao longo dos anos, diversos trabalhos vêm sendo desenvolvidos na área de detecção de intrusão, buscando suprir a necessidade de detecção automática de eventos ilícitos em meio computacional que, por sua, têm se tornado cada vez mais recorrentes e complexos com o passar do tempo. A abordagem por anomalia têm se mostrado promissora, por ser capaz de detectar variações de ataques, contornando uma limitação existente em diversas abordagens preparadas para identificar ataques conhecidos e previamente definidos.

Dentre os diversos métodos empregados para esta tarefa, citam-se métodos estatísticos \cite{jun2001, muzammil2013}, métodos baseados em cálculo de distâncias \cite{zhang2005, syarif2012, aravind2017}, métodos baseados em regras \cite{ilgun1995, roesch1999}, e outros. No contexto deste trabalho, são abordados métodos baseados em algoritmos de inteligência artificial que lidam com o problema de detecção de intrusão em forma de um problema de classificação, sobretudo os que usufruem de bases de dados públicas e amplamente utilizadas na literatura. Deste modo, trabalhos correlacionados serão apresentados a seguir.

\citeasnoun{toosi2007} desempenharam a detecção de intrusão mediante a incorporação de diferentes abordagens baseadas em lógica \textit{fuzzy}, principalmente por meio de classificadores \textit{neuro-fuzzy}. Embora outro classificador, também baseado em \textit{fuzzy} seja utilizado no módulo de decisão final, o principal método considerado para este trabalho trata-se do ANFIS (\textit{Adaptive Neuro-Fuzzy Inference System}).

A base de dados utilizada pelos autores foi a KDD Cup 99, a qual foi previamente abordada no capítulo \ref{Cap:fundamentacao} na seção \ref{Subsec:kdd99} (página \pageref{Subsec:kdd99}), referente às bases de dados comumente utilizadas na literatura. Devido ao grande número de exemplos existentes na base, para fins experimentais, foi utilizada uma amostragem aleatória de 10\% como base de treinamento.

A arquitetura proposta por \citeasnoun{toosi2007} apresenta um classificador ANFIS para cada classe existente na base KDD Cup 99, totalizando cinco (5) diferentes configurações. A justificativa para tal estrutura é de que este método é mais apropriado para classificação binária, embora a solução proposta tenha como objetivo final a classificação de eventos em normal ou intrusivo, especificando a qual classe de ataque o evento pertence. Os resultados demonstraram que a abordagem é promissora e adaptativa, apresentando altas taxas de detecção para classes normais e ataques do tipo DoS. Classes com menos ocorrências na base de dados, como \textit{probe}, U2R e R2L apresentaram piores resultados.

Outro trabalho que utilizou a base KDD Cup 99, como auxílio para avaliação dos modelos construídos, foi desenvolvido por \citeasnoun{sarvari2010}. A abordagem proposta pelos autores também envolve a combinação de diferentes abordagens a fim de detectar ataques, porém os resultados sugerem que existe um limiar onde as taxas de acurácia do sistema se estabilizam, não havendo melhora caso o número de classificadores extrapole o limite observado. Os classificadores desenvolvidos nesta proposta utilizam os métodos KNN, árvore de decisão e diferentes tipos de redes neurais e SVM \cite{mitchell1997}. O estudo apresentado por \citeasnoun{sarvari2010} também busca contornar a limitação de classes desbalanceadas, utilizando os valores de saída dos classificadores desenvolvidos como entrada de uma segunda camada de classificadores. O modelo construído apresentou grande melhora nas taxas de acurácia em todas as classes, quando comparados a outros trabalhos relacionados, mediante o incremento no número de exemplos das classes minoritárias. Os valores de acurácia variaram de 93\% até 99\% para as diferentes classes, exceto para a "U2R", a qual apresentou acurácia de aproximadamente 44\%.

Em 2013, um sistema de detecção de intrusão foi proposto com base em três (3) diferentes algoritmos: K-\textit{Means}, \textit{Neuro-Fuzzy} (ou redes neurais \textit{fuzzy}) e SVM \cite{chandrasekhar2013}. A técnica desenvolvida pelos autores possui quatro (4) fases principais: clusterização mediante aplicação do algoritmo K-\textit{Means} a fim de gerar subconjuntos de treino; submissão dos subconjuntos gerados para redução da dimensão de atributos dos dados por meio de diferentes modelos de rede neural \textit{fuzzy}; aplicação do vetor resultante dos modelos neurais para uma SVM, além da inclusão de um novo atributo baseado nos \textit{clusters} gerados nas etapas anteriores; e, por fim, classificação do evento desempenhada por um SVM do tipo radial. \citeasnoun{chandrasekhar2013} também utilizaram a base KDD Cup 99 como \textit{benchmark} para avaliar o modelo proposto, sendo que este apresentou melhoria significativa em todas as classes de ataques existentes na base. O menor valor de acurácia alcançado foi de 97\%, sendo o único modelo, dentre os outros apresentados no trabalho para fins de comparação, que obteve taxa de acurácia superior a 50\% para todas as classes de ataque.

Conforme apresentado, a vasta maioria dos trabalhos desenvolvidos na área de detecção de intrusão, baseada em métodos de classificação, costuma usufruir da base de dados KDD Cup 99. Entretanto, mediante as diversas análises feitas por pesquisadores ao longo dos anos, constatou-se que esta base possui diversos problemas e limitações. Aspectos como o alto número de exemplos redundantes e duplicados, ou a baixa representatividade do cenário real e atual, haja vista a defasagem dos ataques registrados na base e a evolução das técnicas de invasão e exploração de vulnerabilidades. Este cenário torna questionável a capacidade de detecção de intrusão em ambiente real nos dias atuais, ainda que resultados apresentem altas taxas de acurácia e baixo índice de alarmes falsos \cite{mchugh2000, tavallaee2009, wang2014}.

Diante deste cenário, considera-se o estado da arte em detecção de intrusão trabalhos que buscam pesquisar formas de classificar eventos anômalos mais atuais e mais complexos, ou seja, ao menos utilizam-se de bases atualizadas e com classes de ataques que de fato representem atividades ilícitas modernas, como ataques distribuídos, ou que explorem vulnerabilidades de aplicações Web, dentre outros.

\section{Estado da Arte}

Nos anos 90, redes neurais artificiais mais profundas, ou seja, com maior número de camadas ocultas, estavam em ascensão, e sugeriam grande potencial nas mais diversas áreas de pesquisa que lidavam com problemas complexos. Apesar dos bons resultados, o poder computacional disponível à época inviabilizava diversas aplicações de RNA, principalmente quando comparado ao uso de SVM, abordagem que se mostrava menos custosa. Após, na virada do século, os computadores passaram a ter maior capacidade de processamento e unidades de processamento gráfico (GPU) foram desenvolvidas, possibilitando abordagens de redes neurais a competirem com SVM, visto que, embora ainda fossem mais lentas, apresentavam resultados melhores diante dos mesmos dados de entrada.

Conforme exposto no capítulo anterior, onde são apresentadas algumas das diferentes bases de dados existentes na área de detecção de intrusão, atualmente a CICIDS2017 afigura-se como um conjunto de dados mais relevantes devido aos ataques contidos na mesma, além do ambiente computacional complexo definido para as simulações que a originaram, envolvendo diferentes tipos de dispositivos, sistemas operacionais e topologias de redes \cite{sharafaldin2018}.

\citeasnoun{azwar2018} apresenta um estudo acerca de diferentes técnicas de aprendizado de máquina comumente empregadas na percepção de atividades intrusivas, como diferentes abordagens de árvores de decisão, \textit{Random Forest}, \textit{XGBoost} e redes neurais. Em suas pesquisas, foram levantados diversos \textit{datasets} que se mostraram obsoletos e com limitações referentes a falta de variedade dos tráfegos capturados, dentre outros aspectos. A base CICIDS2017, por sua vez, foi considerada para a avaliação dos modelos construídos. O trabalho publicado por \citeasnoun{azwar2018} não fornece detalhes referentes aos parâmetros definidos para os algoritmos. Entretanto, após realização dos experimentos, o modelo construído obteve acurácia de aproximadamente 92\% na detecção de intrusões.

Também no ano de 2018, foi proposto um sistema de detecção de intrusão baseado no algoritmo \textit{Random Forest} juntamente com abordagem de \textit{Deep Learning} \cite{ustebay2018}. Nesta ocasião, a estratégia de seleção de atributos \textit{Recursive Feature Elimination} (RFE) foi adotada para fins de redução da dimensão da base. O processo de RFE foi desempenhado através do algoritmo baseado em árvores de decisão, \textit{Random Forest}, enquanto a tarefa de classificação dos eventos deu-se através de uma rede neural do tipo \textit{Deep Multilayer Perceptron} (DMLP). Quanto a arquitetura da rede neural treinada, foram utilizadas três camadas ocultas com ReLU como função de ativação. A função de ativação utilizada na camada de saída foi a "\textit{Adaptive Moment Estimation}". Como principais resultados, \citeasnoun{ustebay2018} apresenta uma significativa redução no número de atributos considerados no treinamento do modelo (de 80 para 4, correspondendo a uma redução de 95\% dos atributos), além de acurácia 89\% utilizando apenas os atributos selecionados (diminuição de 2\% quando comparado ao mesmo modelo utilizando todos os atributos disponíveis).

Em \citeasnoun{vinayakumar2019}, os autores sugerem a utilização de \textit{Deep Neural Networks}, ou redes neurais profundas, como classificador de eventos anômalos em redes de computadores. Os pesquisadores ainda ressaltam a importância do desenvolvimento de trabalhos dessa natureza, diante das constantes mudanças de comportamentos de redes de computadores e a evolução acelerada dos \textit{malwares} e ataques desempenhados por ciber-terroristas. Os parâmetros definidos como ótimos para o treinamento da rede neural foram selecionados mediante experimentos na base KDD Cup 99, embora o desempenho da rede neural proposta seja testado em diferentes bases conhecidas, como NSL-KDD, UNSW-NB15, Kyoto, WSN-DS e CICIDS2017, além da própria KDD Cup 99. Após a realização de experimentos variando o número de camadas ocultas que compõem a rede neural, concluiu-se que as redes neurais construídas performaram melhor do que abordagens clássicas como Regressão Logística, \textit{Naive Bayes}, K-\textit{Nearest Neighbors}, Árvores de Decisão, \textit{Random Forest}, e outros.

Com o surgimento da área de IoT (\textit{Internet of Things}), a necessidade de abordagens com menor custo computacional retornaram à tona, visto que os \textit{devices} podem apresentar \textit{hardware} bastante limitado em termos de poder de processamento. Desse modo, a detecção de intrusão em dispositivos de IoT é considerado um dos grandes desafios atuais da área de segurança computacional.

Desse modo, \citeasnoun{jan2019} aborda exclusivamente detecção de ataques de negação de serviço (DoS), uma das classes de ataques que mais ocorrem em ambiente computacional, devido a relativa facilidade de implementação e execução, além do grande impacto quando bem sucedido. Em razão das limitações apresentadas por dispositivos de IoT, como memória reduzida quando comparado a computadores convencionais, e capacidade de bateria, optou-se pelo desenvolvimento de um modelo baseado em \textit{Support Vector Machine} (SVM). Neste trabalho, entretanto, a base CICIDS2017 foi utilizada como auxiliar para a geração de uma outra base, visto que a proposta utiliza um atributo não existente. Resultados apontaram acurácias aproximadas a 98\%, utilizando o conjunto de dados gerados a partir da CICIDS2017 como balizador.